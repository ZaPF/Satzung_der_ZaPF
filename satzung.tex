\documentclass[draft,12pt,oneside]{scrreprt}

% Sprache und Encodings
\usepackage[ngerman]{babel}
\usepackage[T1]{fontenc}
\usepackage[utf8]{inputenc}

% Typographisch interessante Pakete
\usepackage{microtype} % Randkorrektur und andere Anpassungen

% References to Internet and within the document !!!always last package!!!
\usepackage[pdftex,colorlinks=false,
pdftitle={Geschäftsordnung für Plenen der ZaPF},
pdfauthor={Zusammenkunft aller Physikfachschaften},
pdfcreator={pdflatex},
pdfdisplaydoctitle=true]{hyperref}

% Absaetze nicht Einruecken
\setlength{\parindent}{0pt}

\usepackage{draftwatermark}
\SetWatermarkText{Entwurf}
\SetWatermarkScale{5}

% Paragraphen
\renewcommand*\thesection{\S~\arabic{section}.}

\begin{document}

\chapter*{Satzung der ZaPF}

\textbf{Die männliche Anrede gilt im folgenden sowohl für weibliche
als auch für männliche TeilnehmerInnen der ZaPF.}

\section{Name}
Die Tagung der Vertreter der Physik-Fachschaften trägt den Namen Zusammenkunft
aller Physik-Fachschaften, kurz ZaPF.  Sie ist die Nachfolgeorganisation der
Bundes-Fachschaften-Tagung Physik (BuFaTa Physik).

\section{Mitglieder}
Die ZaPF setzt sich aus Vertretern und Mitgliedern der Fachschaften Physik
aller Hochschulen des deutschsprachigen Raumes zusammen.

\section{Aufgaben}
Die ZaPF findet einmal pro Semester statt; sie tagt öffentlich. Sie befasst
sich mit hochschul- und studienrelevanten Themenbereichen.  Die ZaPF dient dem
Sammeln und der Diskussion von Informationen zu diesen Themen und tritt mit
Resultaten gegebenenfalls an die Öffentlichkeit, besitzt aber kein
allgemeinpolitisches Mandat. Des Weiteren dient sie zum Gedanken- und
Ideenaustausch zwischen den Fachschaften. Eine ZaPF beginnt mit dem
Anfangsplenum und endet nach dem Abschlussplenum.

\section{Tagung}
Die ausrichtende Fachschaft legt den Programm-Ablauf der Tagung fest und
erarbeitet ein Protokoll der Veranstaltung, den sogenannten ZaPF-Reader. Sie
stellt davon allen Mitglieds-fachschaften ein Exemplar zur Verfügung.

\section{Organe}
  Die Organe der ZaPF sind
  \begin{enumerate}
    \item \textbf{Das ZaPF-Plenum}\\
          Das ZaPF-Plenum ist das oberste beschlussfassende Gremium der ZaPF
          und setzt sich aus allen Teilnehmern der jeweiligen ZaPF zusammen.
          Einzelne Themen werden in Arbeitskreisen diskutiert und für das
          Plenum vorbereitet. Zu seinen Beschlusskompetenzen zählt auch die
          Wahl der Vertreter für den studentischen Akkreditierungspool für
          Bachelor- und Masterstudiengänge und ähnliche Gremien.  Es beschließt
          ebenfalls die nächsten Veranstaltungsorte der ZaPF.  Den Ablauf der
          Plenen regelt die Geschäftsordnung für Plenen der ZaPF.
    \item \textbf{Der Ständige Ausschuss der Physik-Fachschaften (StAPF)}\\
          Der Ständige Ausschuss der Physik-Fachschaften (StAPF) vertritt die
          ZaPF in der Öffentlichkeit. Der StAPF besteht aus fünf Physik-Studierenden
          von mindestens drei verschiedenen Hochschulen, welche für jeweils ein
          Jahr gewählt werden. Zu jeder im Sommersemester stattfindenden ZaPF
          werden drei Mitglieder des StAPF neu gewählt. Zu jeder im Wintersemester
          stattfindenden ZaPF werden zwei Mitglieder des StAPF neu gewählt.
          Sollten ein oder mehrere Posten im StAPF vakant sein, muss im
          Abschlussplenum der darauf folgenden ZaPF eine Nachbesetzung
          durchgeführt werden. Die Nachbesetzung ist eine Personenwahl wie zur
          Wahl des gesamten StAPF. Sollte es keine Kandidaten für diese Posten
          geben, bleiben sie vakant. Der StAPF besteht aus maximal fünf
          Physik-Studierenden von mindestens drei verschiedenen Hochschulen.
          Er konferiert öffentlich mindestens zweimal zwischen den ZaPFen.
          Termin und Tagungsort (auf einer ZaPF, öffentlicher Chatraum, etc.)
          sind rechtzeitig an geeigneter Stelle bekannt zu machen. Der StAPF ist
          an die Weisungen des Plenums gebunden, kann jedoch eigenverantwortlich
          handeln und muss seine Beschlüsse dem ZaPF-Plenum gegenüber vertreten.
          Die Entscheidungen innerhalb des StAPF müssen in diesen Fällen einstimmig
          fallen. Der StAPF gibt Informationen umgehend an die Fachschaften weiter.
          Auf jeder ZaPF ist darüber hinaus ein Rechenschaftsbericht vorzulegen.
          Der StAPF ist für die Archivierung und Veröffentlichung der Ergebnisse
          der ZaPF verantwortlich, des Weiteren ist er Unterzeichner der
          ZaPF-Veröffentlichungen. Der StAPF wählt sich aus seiner Mitte einen
          Sprecher. Sollte kein StAPF gewählt werden übernimmt das Plenum der
          ZaPF die Aufgaben des StAPF.
  \end{enumerate}

\section{Satzungsänderungen}
Änderungen dieser Satzung benötigen eine 2/3-Mehrheit, wobei Beschlussfähigkeit
des Plenums vor der Abstimmung zwingend festzustellen ist. Satzungsänderungen
sind nicht durch Initiativanträge möglich und können nur auf dem Endplenum
abgestimmt werden. Wünsche nach einer Satzungsänderung sind bis spätestens
sieben Tage vor dem Anfangsplenum geeignet (z.B. über die ZaPF-Mailingliste)
zusammen mit einem Antragsentwurf oder mindestens einer schriftlichen
Begründung und einem konkreten Thema der Satzungsänderung anzukündigen. Auf der
ZaPF muss dann zwingend ein Arbeitskreis zum Thema der vorgeschlagenen
Satzungsänderungen durchgeführt werden, dessen
Satzungsänderungsantrag/Satzungsänderungsanträge bis spätestens 15:00 Uhr am
Vortag des Endplenums bei der die ZaPF ausrichtenden Fachschaft eingereicht und
ausgehängt werden müssen.

\section*{Schlussbestimmungen und Änderungshistorie}
Die vorliegende Satzung wurde anlässlich der ZaPF '06 in Dresden vorbereitet,
mit einer Zweidrittelmehrheit der anwesenden Fachschaften beschlossen und
angenommen. Diese Satzung setzt alle bisherigen außer Kraft. Sie trat zum
28.05.2006 in Kraft.

Inhaltliche Änderungen wurden vorgenommen auf der:

\begin{itemize}

  \item Sommer-ZaPF 2007 in Berlin

  \item Sommer-ZaPF 2008 in Konstanz

  \item Sommer-ZaPF 2009 in Göttingen

  \item Sommer-ZaPF 2011 in Dresden

  \item Sommer-ZaPF 2013 in Jena

\end{itemize}

\end{document}
