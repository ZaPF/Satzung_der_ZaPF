\documentclass[%
    ngerman,
   %draft,                  % Entwurf draft bedeutet draft=true, wird aber auch von graphicx verstanden
    final,
    twoside=true,           % zweiseitig
    DIV=14,                 % Satzspiegelberechnugn
    BCOR=12mm,              % Bindekorrektur
    fontsize=12pt,          % Schriftgröße
    paper=a4,               % Papierformat
    %toc=listof,             % Listen in das Inhaltsverzeichnis
    %toc=bib,                % Literatuverzeichnis ins Inhaltsverzeichnis
    headsepline=false,       %keine Trennlinie zwischen Kopf und Text
    footsepline=false,      % keine Linie im Dokumentfuss
    footnotes=multiple,     % Fußnoten werden getrennt
    abstract=false         % Zusammenfassung mit Wort "Zusammenfassung" wird selbstübernommen!
]{scrreprt}

%Design and Language Definition
    \usepackage{ngerman}
    \usepackage[T1]{fontenc}
    \usepackage[utf8]{inputenc}
    \usepackage[automark]{scrpage2} %Header definiton for Koma Script Classes

% Graphic Input and Styling
    \usepackage{xspace} %fügt hinter Makros einen entsprechenden Space ein wenn nötig oder eben nciht \xspace


%Special Stylings

    \usepackage{url}

%Pakete für die Quellcode Listings

%Typographisch interessante Pakete
    \usepackage{mathpazo}   %Palatino
    \linespread{1.05}       %Palatino braucht einen höheren Durchschuss
    \renewcommand{\sfdefault}{uop} %Optima clone classico als Überschrift
    % Siehe zur Installation von classico die Geschaeftsordnungs-Header Datei.
    \usepackage[tracking=true]{microtype} % Randkorrektur und andere Anpassungen
    \DeclareMicrotypeSet*[tracking]{my}   % Sperrt Kapitälchen
      { font = */*/*/sc/* }%
    \SetTracking{ encoding = *, shape = sc }{ 45 }%
    \KOMAoptions{DIV=last} % Satzspiegel neu berechnen, da Paladino als Schrift gewählt


%References to Internet and within the document !!!always last package!!!
    \usepackage[
        pdftex,
    %   % Farben fuer die Links
       colorlinks=true,         % Links erhalten Farben statt Kaeten
       urlcolor=cyan,    % \href{...}{...} external (URL)
       filecolor=pdffilecolor,  % \href{...} local file
       linkcolor=red,  %\ref{...} and \pageref{...}
       citecolor=green,  %
    %   % Links
    %      raiselinks=true,             % calculate real height of the link
    %   breaklinks,              % Links berstehen Zeilenumbruch
    %   backref=page,            % Backlinks im Literaturverzeichnis (section, slide, page, none)
    %   pagebackref=true,        % Backlinks im Literaturverzeichnis mit Seitenangabe
    %   verbose,
    %   hyperindex=true,         % backlinkex index
    %      linktocpage=true,        % Inhaltsverzeichnis verlinkt Seiten
    %   hyperfootnotes=false,     % Keine Links auf Fussnoten
    %   % Bookmarks
       bookmarks=true,          % Erzeugung von Bookmarks fuer PDF-Viewer
       bookmarksopenlevel=-1,    % Gliederungstiefe der Bookmarks
    %   bookmarksopen=true,      % Expandierte Untermenues in Bookmarks
    %   bookmarksnumbered=true,  % Nummerierung der Bookmarks
       bookmarkstype=toc,       % Art der Verzeichnisses
    %   % Anchors
    %   plainpages=false,        % Anchors even on plain pages ?
    %   pageanchor=true,         % Pages are linkable
    %   % PDF Informationen
     	 pdftitle={Satzung der Zusammenkunft aller Physik Fachschaften (ZaPF)},             % Titel
       pdfauthor={AK Satzung der ZaPF },    
       pdfcreator={PDFLaTeX, hyperref, KOMA-Script},        % Ersteller
       pdfproducer={pdfTeX 1.40.10},   %Produzent
       pdftoolbar=true,         % Shows PDFToolbar
       pdfdisplaydoctitle=true, % Dokumententitel statt Dateiname im Fenstertitel
       pdfstartview=FitV,       % Dokument wird Fit Vertical geoeffnet
       pdfpagemode=UseOutlines, % Bookmarks im Viewer anzeigen
    %   pdfpagelabels=true,           % set PDF page labels
       pdfpagelayout=TwoPageRight%, % zweiseitige Darstellung: ungerade Seiten
    %                                        % rechts im PDF-Viewer
    %   %pdfpagelayout=SinglePage, % einseitige Darstellung
    ]{hyperref}




%Special Commands

\renewcommand \thesection {\Roman{section}}
\setlength{\parindent}{0pt}


\begin{document}

\chapter*{Satzung der ZaPF}

\textbf{Die männliche Anrede gilt im folgenden sowohl für weibliche
als auch für männliche
 TeilnehmerInnen der ZaPF.}

\section*{§ 1 Name}
Die Tagung der Vertreter der Physik-Fachschaften trägt den Namen Zusammenkunft aller
Physik-Fachschaften (ZaPF). Sie ist die Nachfolgeorganisation der Bundes-Fachschaften-Tagung Physik
(BuFaTa Physik).

\section*{§ 2 Mitglieder}
Die ZaPF setzt sich aus Vertretern und Mitgliedern der Fachschaften Physik aller Hochschulen
des deutschsprachigen Raumes zusammen.

\section*{§ 3 Aufgaben}
Die ZaPF findet einmal pro Semester statt; sie tagt öffentlich. Sie befasst sich mit
hochschul- und studienrelevanten Themenbereichen. Die ZaPF dient dem Sammeln und der Diskussion von Informationen
zu diesen Themen und tritt mit Resultaten gegebenenfalls an die Öffentlichkeit, besitzt aber kein allgemeinpolitisches
Mandat. Des Weiteren dient sie zum Gedanken- und Ideenaustausch zwischen den Fachschaften. Alles Weitere regelt die
Geschäftsordnung für die Plenen der ZaPF analog.

\section*{§ 4 Tagung}
Die ausrichtende Fachschaft legt den Programm-Ablauf der Tagung fest und erarbeitet ein Protokoll der Veranstaltung,
den sogenannten ZaPF-Reader. Sie stellt davon allen Mitglieds-fachschaften ein Exemplar zur Verfügung.
\newpage
\section*{§ 5 Organe}
Die Organe der ZaPF sind
\begin{enumerate}
\item{\textbf{Das ZaPF-Plenum}\\
Das ZaPF-Plenum ist das oberste beschlussfassende Gremium der ZaPF und setzt sich aus allen Teilnehmern der jeweiligen ZaPF
zusammen. Einzelne Themen werden in Arbeits-kreisen diskutiert und für das Plenum vorbereitet. Zu seinen Beschlusskompetenzen
zählt auch die Wahl der Vertreter für den studentischen Akkreditierungspool für Bachelor- und Masterstudiengänge und ähnliche
Gremien. Es beschließt ebenfalls die nächsten Veranstaltungsorte der ZaPF. Alles weitere regelt die Geschäftsordnung für die Plenen der
ZaPF analog.}
\item{\textbf{Der Ständige Ausschuss der Physik-Fachschaften (StAPF)}\\
Der Ständige Ausschuss der Physik-Fachschaften (StAPF) vertritt die
ZaPF in der Öffentlichkeit. Der StAPF wird auf jeder ZaPF neu
gewählt. Er besteht aus maximal fünf Physik-Studierenden von
mindestens drei verschiedenen Hochschulen. Er konferiert öffentlich
mindestens zweimal im Semester. Termin und Tagungsort (auf
einer ZaPF, öffentlicher Chatraum, etc.) sind rechtzeitig an
geeigneter Stelle bekannt zu machen. Der StAPF ist an die Weisungen
des Plenums gebunden, kann jedoch eigenverantwortlich handeln und
muss seine Beschlüsse dem ZaPF-Plenum gegenüber vertreten. Die
Entscheidungen innerhalb des StAPF müssen in diesen Fällen
einstimmig fallen. Der StAPF gibt Informationen umgehend an die
Fachschaften weiter. Auf jeder ZaPF ist darüber hinaus ein
Rechenschaftsbericht vorzulegen. Der StAPF ist für die Archivierung
und Veröffentlichung der Ergebnisse der ZaPF verantwortlich, des
Weiteren ist er Unterzeichner der ZaPF-Veröffentlichungen. Der StAPF
wählt sich aus seiner Mitte einen Sprecher.}
\end{enumerate}
\section*{§ 6 Wahlen}
Wahlen und Abstimmungen regelt die Geschäftsordnung für die Plenen der ZaPF analog.

\section*{§ 7 Satzungsänderungen}
Eine Änderung der Satzung entspricht dem Vorgehen bei Änderung der Geschäftsordnung für die Plenen der ZaPF.
Die Mehrheiten und Fristen sind entsprechend zu wahren.

\section*{Schlussbestimmungen}
Die vorliegende Satzung wurde anlässlich der ZaPF '06 in Dresden
vorbereitet, mit einer Zweidrittelmehrheit der anwesenden
Fachschaften beschlossen und angenommen. Diese Satzung setzt alle
bisherigen außer Kraft. Sie tritt zum 28.05.2006 in Kraft.

\end{document}
% Beschloßen in Dresden,geändert in Berlin
