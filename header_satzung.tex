\documentclass[%
    ngerman,
   %draft,                  % Entwurf draft bedeutet draft=true, wird aber auch von graphicx verstanden
    final,
    twoside=true,           % zweiseitig
    DIV=14,                 % Satzspiegelberechnugn
    BCOR=12mm,              % Bindekorrektur
    fontsize=12pt,          % Schriftgröße
    paper=a4,               % Papierformat
    %toc=listof,             % Listen in das Inhaltsverzeichnis
    %toc=bib,                % Literatuverzeichnis ins Inhaltsverzeichnis
    headsepline=false,       %keine Trennlinie zwischen Kopf und Text
    footsepline=false,      % keine Linie im Dokumentfuss
    footnotes=multiple,     % Fußnoten werden getrennt
    abstract=false         % Zusammenfassung mit Wort "Zusammenfassung" wird selbstübernommen!
]{scrreprt}

%Design and Language Definition
    \usepackage{ngerman}
    \usepackage[T1]{fontenc}
    \usepackage[utf8]{inputenc}
    \usepackage[automark]{scrpage2} %Header definiton for Koma Script Classes

% Graphic Input and Styling
    \usepackage{xspace} %fügt hinter Makros einen entsprechenden Space ein wenn nötig oder eben nciht \xspace


%Special Stylings

    \usepackage{url}

%Pakete für die Quellcode Listings

%Typographisch interessante Pakete
    \usepackage{mathpazo}   %Palatino
    \linespread{1.05}       %Palatino braucht einen höheren Durchschuss
    \renewcommand{\sfdefault}{uop} %Optima clone classico als Überschrift
    % Siehe zur Installation von classico die Geschaeftsordnungs-Header Datei.
    \usepackage[tracking=true]{microtype} % Randkorrektur und andere Anpassungen
    \DeclareMicrotypeSet*[tracking]{my}   % Sperrt Kapitälchen
      { font = */*/*/sc/* }%
    \SetTracking{ encoding = *, shape = sc }{ 45 }%
    \KOMAoptions{DIV=last} % Satzspiegel neu berechnen, da Paladino als Schrift gewählt


%References to Internet and within the document !!!always last package!!!
    \usepackage[
        pdftex,
    %   % Farben fuer die Links
       colorlinks=true,         % Links erhalten Farben statt Kaeten
       urlcolor=cyan,    % \href{...}{...} external (URL)
       filecolor=pdffilecolor,  % \href{...} local file
       linkcolor=red,  %\ref{...} and \pageref{...}
       citecolor=green,  %
    %   % Links
    %      raiselinks=true,             % calculate real height of the link
    %   breaklinks,              % Links berstehen Zeilenumbruch
    %   backref=page,            % Backlinks im Literaturverzeichnis (section, slide, page, none)
    %   pagebackref=true,        % Backlinks im Literaturverzeichnis mit Seitenangabe
    %   verbose,
    %   hyperindex=true,         % backlinkex index
    %      linktocpage=true,        % Inhaltsverzeichnis verlinkt Seiten
    %   hyperfootnotes=false,     % Keine Links auf Fussnoten
    %   % Bookmarks
       bookmarks=true,          % Erzeugung von Bookmarks fuer PDF-Viewer
       bookmarksopenlevel=-1,    % Gliederungstiefe der Bookmarks
    %   bookmarksopen=true,      % Expandierte Untermenues in Bookmarks
    %   bookmarksnumbered=true,  % Nummerierung der Bookmarks
       bookmarkstype=toc,       % Art der Verzeichnisses
    %   % Anchors
    %   plainpages=false,        % Anchors even on plain pages ?
    %   pageanchor=true,         % Pages are linkable
    %   % PDF Informationen
     	 pdftitle={Satzung der Zusammenkunft aller Physik Fachschaften (ZaPF)},             % Titel
       pdfauthor={AK Satzung der ZaPF },    
       pdfcreator={PDFLaTeX, hyperref, KOMA-Script},        % Ersteller
       pdfproducer={pdfTeX 1.40.10},   %Produzent
       pdftoolbar=true,         % Shows PDFToolbar
       pdfdisplaydoctitle=true, % Dokumententitel statt Dateiname im Fenstertitel
       pdfstartview=FitV,       % Dokument wird Fit Vertical geoeffnet
       pdfpagemode=UseOutlines, % Bookmarks im Viewer anzeigen
    %   pdfpagelabels=true,           % set PDF page labels
       pdfpagelayout=TwoPageRight%, % zweiseitige Darstellung: ungerade Seiten
    %                                        % rechts im PDF-Viewer
    %   %pdfpagelayout=SinglePage, % einseitige Darstellung
    ]{hyperref}




%Special Commands

\renewcommand \thesection {\Roman{section}}
\setlength{\parindent}{0pt}
