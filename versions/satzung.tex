\documentclass[12pt,oneside]{scrartcl}
% generated by Docutils <http://docutils.sourceforge.net/>
\usepackage{fixltx2e} % LaTeX patches, \textsubscript
\usepackage{cmap} % fix search and cut-and-paste in Acrobat
\usepackage{ifthen}
\usepackage[T1]{fontenc}
\usepackage[utf8]{inputenc}
\setcounter{secnumdepth}{0}
\usepackage{textcomp} % text symbol macros

%%% Custom LaTeX preamble


%%% User specified packages and stylesheets
% embedded stylesheet: ./util/preamble.tex
% Sprache und Encodings
\usepackage[ngerman]{babel}

% Typographisch interessante Pakete
\usepackage{microtype} % Randkorrektur und andere Anpassungen

% References to Internet and within the document !!!always last package!!!
\usepackage[pdftex,colorlinks=false,
pdftitle={Geschäftsordnung für Plenen der ZaPF},
pdfauthor={Zusammenkunft aller Physikfachschaften},
pdfcreator={pdflatex},
pdfdisplaydoctitle=true]{hyperref}

% Absaetze nicht Einruecken
\setlength{\parindent}{0pt}

\renewcommand*\thesection{\arabic{section}}


%%% Fallback definitions for Docutils-specific commands

% hyperlinks:
\ifthenelse{\isundefined{\hypersetup}}{
  \usepackage[colorlinks=true,linkcolor=blue,urlcolor=blue]{hyperref}
  \urlstyle{same} % normal text font (alternatives: tt, rm, sf)
}{}
\hypersetup{
  pdftitle={Satzung der ZaPF},
}

%%% Title Data
\title{\phantomsection%
  Satzung der ZaPF%
  \label{satzung-der-zapf}}
\author{}
\date{}

%%% Body
\begin{document}
\maketitle


\section{§1 Name%
  \label{name}%
}

Die Tagung der Vertreterinnen und Vertreter der Physik-Fachschaften trägt den
Namen Zusammenkunft aller Physik-Fachschaften, kurz ZaPF.
Sie ist die Nachfolgeorganisation der Bundes-Fachschaften-Tagung Physik (BuFaTa Physik).


\section{§2 Mitglieder%
  \label{mitglieder}%
}

Die ZaPF setzt sich aus Vertretern und Vertreterinnen und Mitgliedern der
Fachschaften Physik aller Hochschulen des deutschsprachigen Raumes zusammen.


\section{§3 Aufgaben%
  \label{aufgaben}%
}

Die ZaPF findet einmal pro Semester statt; sie tagt öffentlich. Sie befasst
sich mit hochschul- und studienrelevanten Themenbereichen.

Die ZaPF dient dem Sammeln und der Diskussion von Informationen zu diesen Themen
und tritt mit Resultaten gegebenenfalls an die Öffentlichkeit, besitzt aber kein
allgemeinpolitisches Mandat.
Des Weiteren dient sie zum Gedanken- und Ideenaustausch zwischen den Fachschaften.


\section{§4 Tagung%
  \label{tagung}%
}

Die ausrichtende Fachschaft legt den Programm-Ablauf der Tagung fest und
erarbeitet ein Protokoll der Veranstaltung, den sogenannten ZaPF-Reader. Sie
stellt davon allen Mitglieds-fachschaften ein Exemplar zur Verfügung.

Die Tagung beginnt mit dem Anfangsplenum und endet nach dem Abschlussplenum.


\section{§5 Organe%
  \label{organe}%
}

Die Organe der ZaPF sind das ZaPF-Plenum, der Ständige Ausschuss der
Physik-Fachschaften (StAPF), die Vertrauenspersonen und das Kommunikationsgremium.


\subsection{(a) Das ZaPF-Plenum%
  \label{a-das-zapf-plenum}%
}

Das ZaPF-Plenum ist das oberste beschlussfassende Gremium der ZaPF und setzt
sich aus allen Teilnehmerinnen und Teilnehmern der jeweiligen ZaPF zusammen.

Einzelne Themen werden in Arbeitskreisen diskutiert und für das Plenum vorbereitet.
Zu seinen Beschlusskompetenzen zählt auch die Wahl der Vertreter und Vertreterinnern
für den studentischen Akkreditierungspool für Bachelor- und Masterstudiengänge und
ähnliche Gremien.

Das Plenum beschließt ebenfalls die nächsten Veranstaltungsorte der ZaPF.

Den Ablauf der Plenen regelt die Geschäftsordnung für Plenen der ZaPF.


\subsection{(b) Der Ständige Ausschuss der Physik-Fachschaften%
  \label{b-der-standige-ausschuss-der-physik-fachschaften}%
}

Der Ständige Ausschuss der Physik-Fachschaften (StAPF) vertritt die ZaPF in der
Öffentlichkeit.

Der StAPF besteht aus maximal fünf Physik-Studierenden von mindestens drei
verschiedenen Hochschulen, welche für jeweils ein Jahr gewählt werden.

Zu jeder im Sommersemester stattfindenden ZaPF werden drei Mitglieder des StAPF
neu gewählt.
Zu jeder im Wintersemester stattfindenden ZaPF werden zwei Mitglieder des StAPF
neu gewählt.

Sollten ein oder mehrere Posten im StAPF vakant sein, muss im Abschlussplenum der
darauf folgenden ZaPF eine Nachbesetzung durchgeführt werden.
Die nachbesetzte Person bleibt für die Restdauer der Wahlperiode des
ausgeschiedenen Mitgliedes im Amt.
Die Nachbesetzung ist eine Personenwahl wie zur Wahl des gesamten StAPF.
Sollte es keine Kandidatinnen oder Kandidaten für diese Posten geben, bleiben
sie vakant.

Der StAPF konferiert öffentlich mindestens zweimal zwischen den ZaPFen.
Termin und Tagungsort (auf einer ZaPF, öffentlicher Chatraum, etc.) sind
rechtzeitig an geeigneter Stelle bekannt zu machen.

Der StAPF ist an die Weisungen des Plenums gebunden, kann jedoch
eigenverantwortlich handeln und muss seine Beschlüsse dem ZaPF-Plenum gegenüber
vertreten.
Die Entscheidungen innerhalb des StAPF müssen in diesen Fällen einstimmig fallen.

Der StAPF gibt Informationen umgehend an die Fachschaften weiter.
Auf jeder ZaPF ist darüber hinaus ein Rechenschaftsbericht vorzulegen.

Der StAPF ist für die Archivierung und Veröffentlichung der Ergebnisse der ZaPF
verantwortlich, des Weiteren ist er Unterzeichner der ZaPF-Veröffentlichungen.
Der StAPF wählt sich aus seiner Mitte einen Sprecher.

Sollte kein StAPF gewählt werden übernimmt das Plenum der ZaPF die Aufgaben
des StAPF.


\subsection{(c) Die Vertrauenspersonen%
  \label{c-die-vertrauenspersonen}%
}

Die Vertrauenspersonen dienen als Anlaufstelle für hilfesuchende Personen, die
Ausgrenzung, Diskriminierung oder Belästigung im Rahmen der ZaPF erfahren haben.

Die Wahl der höchstens sechs Vertrauenspersonen ist zu Beginn jeder ZaPF durchzuführen.
Ihre Amtszeit endet mit dem Ende des Abschlussplenums oder Niederlegung des Amtes.

Die Wahl der Vertrauenspersonen ist in der Geschäftsordnung für Plenen der ZaPF
gesondert zu regeln.


\subsection{(d) Das Kommunikationsgremium%
  \label{d-das-kommunikationsgremium}%
}

Das Kommunikationsgremium ist ein gemeinsames Gremium von ZaPF und jDPG.

Die Aufgaben dieses Gremiums sind der Austausch zwischen ZaPF und jDPG sowie
die Veröffentlichung gemeinsamer Beschlüsse.
Weiterhin entsendet dieses Gremium einen gemeinsamen Vertreter oder eine
gemeinsame Vertreterin zur KFP.

Die ZaPF und jDPG entsenden je ein Mitglied in das Kommunikationsgremium.

Die Amtszeit der Mitglieder im Kommunikationsgremium beläuft sich auf ein Jahr.

Näheres regelt das Dokument zur \textquotedbl{}Regelung zur Zusammenarbeit von jDPG und ZaPF
in hochschulpolitischen Fragestellungen\textquotedbl{} in der Fassung vom Endplenum der ZaPF
im Sommersemester 2010 in Frankfurt.


\section{§6 Satzungsänderungen%
  \label{satzungsanderungen}%
}

Änderungen dieser Satzung benötigen eine Zweidrittelmehrheit, wobei Beschlussfähigkeit
des Plenums vor der Abstimmung zwingend festzustellen ist. Satzungsänderungen
sind nicht durch Initiativanträge möglich und können nur auf dem Endplenum
abgestimmt werden.

Wünsche nach einer Satzungsänderung sind bis spätestens sieben Tage vor dem
Anfangsplenum geeignet (z.B. über die ZaPF-Mailingliste)
zusammen mit einem Antragsentwurf oder mindestens einer schriftlichen
Begründung und einem konkreten Thema der Satzungsänderung anzukündigen.

Auf der ZaPF muss dann zwingend ein Arbeitskreis zum Thema der vorgeschlagenen
Satzungsänderungen durchgeführt werden, dessen Satzungsänderungsantrag bzw.
Satzungsänderungsanträge bis spätestens 15:00 Uhr am Vortag des Endplenums bei
der die ZaPF ausrichtenden Fachschaft eingereicht und ausgehängt werden müssen.


\section{Schlussbestimmungen und Änderungshistorie%
  \label{schlussbestimmungen-und-anderungshistorie}%
}

Die vorliegende Satzung wurde anlässlich der ZaPF '06 in Dresden vorbereitet,
mit einer Zweidrittelmehrheit der anwesenden Fachschaften beschlossen und
angenommen. Diese Satzung setzt alle bisherigen außer Kraft. Sie trat zum
28.05.2006 in Kraft.

Inhaltliche Änderungen wurden vorgenommen auf der:
%
\begin{itemize}

\item Sommer-ZaPF 2007 in Berlin,

\item Sommer-ZaPF 2008 in Konstanz,

\item Sommer-ZaPF 2009 in Göttingen,

\item Sommer-ZaPF 2011 in Dresden,

\item Sommer-ZaPF 2013 in Jena,

\item Sommer-ZaPF 2014 in Düsseldorf,

\item und auf Winter-ZaPF 2014 in Bremen.

\end{itemize}

\end{document}
