% Options for packages loaded elsewhere
\PassOptionsToPackage{unicode,pdfcreator=\{pdflatex\},pdfdisplaydoctitle=true}{hyperref}
\PassOptionsToPackage{hyphens}{url}
%
\documentclass[
  a4paper,
  oneside]{scrartcl}
\usepackage{amsmath,amssymb}
\usepackage{iftex}
\ifPDFTeX
  \usepackage[T1]{fontenc}
  \usepackage[utf8]{inputenc}
  \usepackage{textcomp} % provide euro and other symbols
\else % if luatex or xetex
  \usepackage{unicode-math} % this also loads fontspec
  \defaultfontfeatures{Scale=MatchLowercase}
  \defaultfontfeatures[\rmfamily]{Ligatures=TeX,Scale=1}
\fi
\usepackage{lmodern}
\ifPDFTeX\else
  % xetex/luatex font selection
\fi
% Use upquote if available, for straight quotes in verbatim environments
\IfFileExists{upquote.sty}{\usepackage{upquote}}{}
\IfFileExists{microtype.sty}{% use microtype if available
  \usepackage[]{microtype}
  \UseMicrotypeSet[protrusion]{basicmath} % disable protrusion for tt fonts
}{}
\makeatletter
\@ifundefined{KOMAClassName}{% if non-KOMA class
  \IfFileExists{parskip.sty}{%
    \usepackage{parskip}
  }{% else
    \setlength{\parindent}{0pt}
    \setlength{\parskip}{6pt plus 2pt minus 1pt}}
}{% if KOMA class
  \KOMAoptions{parskip=half}}
\makeatother
\usepackage{xcolor}
\setlength{\emergencystretch}{3em} % prevent overfull lines
\providecommand{\tightlist}{%
  \setlength{\itemsep}{0pt}\setlength{\parskip}{0pt}}
\setcounter{secnumdepth}{5}
\ifLuaTeX
\usepackage[bidi=basic]{babel}
\else
\usepackage[bidi=default]{babel}
\fi
\babelprovide[main,import]{ngerman}
% get rid of language-specific shorthands (see #6817):
\let\LanguageShortHands\languageshorthands
\def\languageshorthands#1{}
\ifLuaTeX
  \usepackage{selnolig}  % disable illegal ligatures
\fi
\usepackage{bookmark}
\IfFileExists{xurl.sty}{\usepackage{xurl}}{} % add URL line breaks if available
\urlstyle{same}
\hypersetup{
  pdftitle={Satzung der ZaPF},
  pdfauthor={Zusammenkunft aller Physikfachschaften},
  pdflang={de-DE},
  hidelinks,
  pdfcreator={LaTeX via pandoc}}

\title{Satzung der ZaPF}
\author{Zusammenkunft aller Physikfachschaften}
\date{}

\begin{document}
\maketitle

\section{Name}\label{name}

Die Tagung der vertretenden Personen der Physik-Fachschaften trägt den
Namen Zusammenkunft aller Physik-Fachschaften, kurz ZaPF. Sie ist die
Nachfolgeorganisation der Bundes-Fachschaften-Tagung Physik (BuFaTa
Physik).

\section{Selbstverständnis}\label{selbstverstuxe4ndnis}

Die ZaPF versteht sich als antifaschistisch, demokratisch und offen für
alle Menschen, welche diese Grundsätze teilen.

\section{Mitglieder}\label{mitglieder}

Die ZaPF setzt sich aus vertretenden Personen und Mitgliedern der
Fachschaften Physik aller Hochschulen des deutschsprachigen Raumes
zusammen.

Die Mitglieder verpflichten sich zur Einhaltung des Verhaltenskodex der
ZaPF.

\section{Aufgaben}\label{aufgaben}

Die ZaPF findet einmal pro Semester statt und tagt öffentlich. Sie dient
dem Sammeln und der Diskussion von Informationen und tritt mit den
Resultaten gegebenenfalls an die Öffentlichkeit oder an Dritte heran.
Des Weiteren dient sie zum Gedanken- und Ideenaustausch zwischen den
Fachschaften.

Die ZaPF befasst sich mit studien- und hochschulrelevanten Themen. Sie
besitzt kein allgemeinpolitisches Mandat, kann sich jedoch in Bezug auf
hochschulpolitische Themen auch allgemeinpolitisch äußern. Hierbei muss
ein Zusammenhang zu studien- und hochschulpolitischen Belangen
unmittelbar bestehen und deutlich erkennbar bleiben.

\section{Tagung}\label{tagung}

Die ausrichtende Fachschaft legt den Programmablauf der Tagung fest und
erarbeitet ein Protokoll der Veranstaltung, den sogenannten ZaPF-Reader.
Sie stellt davon allen Mitgliedsfachschaften ein Exemplar zur Verfügung.

Die Tagung beginnt mit dem Anfangsplenum und endet nach dem
Abschlussplenum.

\section{Organe}\label{organe}

Die Organe der ZaPF sind das ZaPF-Plenum, der Ständige Ausschuss der
Physik-Fachschaften (StAPF), das Awarenessgremium, die
Vertrauenspersonen, das Kommunikationsgremium (KomGrem) und der
Technische Organisationsausschuss aller Physikfachschaften (TOPF).

Die Wahlen von Mitgliedern des StAPF, des Awarenessgremiums, des KomGrem
und des TOPF sind Personenwahlen entsprechend der Geschäftsordnung der
ZaPF.

Die Mitgliedschaft im StAPF, dem Awarenessgremium, der Gruppe der
Vertrauenspersonen, dem Kommunikationsgremium oder dem TOPF endet mit
Ablauf der Amtszeit, Ableben der amtsinhabenden Person, Niederlegung des
Amtes oder Abwahl mit Zweidrittelmehrheit durch das Plenum. Der Antrag
auf Abwahl ist bis 15:00 Uhr am Vortag bei der ausrichtenden Fachschaft
anzukündigen.

Bis zur Nachwahl bleibt ein unbesetztes Amt vakant. Bei der Nachwahl
wird das Amt bis zum Ablauf der Restdauer der Amtszeit besetzt. Die
Nachwahl findet auf einem Plenum der nächstmöglichen Tagung statt.
Sollten nach einer Wahl Posten unbesetzt sein, bleiben sie vakant.

Falls mindestens zwei Drittel der Mitglieder eines Gremiums das Amt
niederlegen, gelten auch die Ämter der übrigen Mitglieder dieses
Gremiums als vakant.

Für Amtszeiten sei ein Jahr definiert als die Zeit zwischen einer
Sommer-ZaPF und der ihr nächsten nachfolgenden Sommer-ZaPF bzw. einer
Winter-ZaPF und der ihr nächsten nachfolgenden Winter-ZaPF. Das Jahr
beginnt mit dem Ende des Plenums in dem die Wahl für ein Organ
turnusmäßig stattfindet. Es endet mit dem Plenum, in dem die Wahl zur
Neubesetzung der entsprechenden Plätze des Organs turnusmäßig
stattfindet, spätestens jedoch mit dem Ende der Tagung.

\subsection{Das ZaPF-Plenum}\label{das-zapf-plenum}

Das ZaPF-Plenum ist das oberste beschlussfassende Gremium der ZaPF und
setzt sich aus allen teilnehmenden Personen der jeweiligen ZaPF
zusammen.

Einzelne Themen werden in Arbeitskreisen diskutiert und für das Plenum
vorbereitet. Zu seinen Beschlusskompetenzen zählt auch die Wahl der
vertretenden Personen für den studentischen Akkreditierungspool für
Bachelor- und Masterstudiengänge und ähnliche Gremien.

Das Plenum beschließt ebenfalls die nächsten Veranstaltungsorte der ZaPF
und den Verhaltenskodex der ZaPF.

Den Ablauf der Plenen regelt die Geschäftsordnung für Plenen der ZaPF.

\subsection{Der Ständige Ausschuss der
Physik-Fachschaften}\label{der-stuxe4ndige-ausschuss-der-physik-fachschaften}

Der Ständige Ausschuss der Physik-Fachschaften (StAPF) vertritt die ZaPF
in der Öffentlichkeit.

Der StAPF besteht aus maximal fünf natürlichen Personen von mindestens
drei verschiedenen Hochschulen, welche für jeweils ein Jahr gewählt
werden.

Die Amtszeit von drei Mitgliedern des StAPF beginnt zu einer im
Sommersemester stattfindenden ZaPF und die zweier StAPF-Mitglieder zu
einer im Wintersemester stattfindenden ZaPF.

Der StAPF konferiert öffentlich mindestens zweimal zwischen den ZaPFen.
Termin und Tagungsort (auf einer ZaPF, öffentlicher Chatraum, etc.) sind
rechtzeitig an geeigneter Stelle bekannt zu machen.

Der StAPF ist an die Weisungen des Plenums gebunden, kann jedoch
eigenverantwortlich handeln und muss seine Beschlüsse dem ZaPF-Plenum
gegenüber vertreten. Die Entscheidungen innerhalb des StAPF müssen in
diesen Fällen einstimmig fallen. Der StAPF ist beschlussfähig falls
mindestens drei seiner Mitglieder auf einer Sitzung anwesend sind und
der Beschluss in der Sitzungseinladung angekündigt wurde.

Der StAPF gibt Informationen umgehend an die Fachschaften weiter. Auf
jeder ZaPF ist darüber hinaus ein Rechenschaftsbericht vorzulegen.

Der StAPF ist für die Archivierung und Veröffentlichung der Ergebnisse
der ZaPF verantwortlich, des Weiteren ist er Unterzeichner der
ZaPF-Veröffentlichungen. Der StAPF wählt sich aus seiner Mitte eine ihn
repräsentierende Person. Diese Person darf sich als die ``STIMME der
ZaPF'' bezeichnen.

Sollten alle Posten des StAPFes vakant sein, übernehmen die von der ZaPF
entsandten Mitglieder des Kommunikationsgremiums oder, falls diese
vakant sind, die Mitglieder des Technischen Organisationsausschuss aller
Physikfachschaften oder, falls auch diese vakant sind, die Mitglieder
der letzten die ZaPF ausrichtenden Fachschaft die Archivierungs- und
Veröffentlichungsaufgaben des StAPF.

\subsection{Die Vertrauenspersonen}\label{die-vertrauenspersonen}

Die Vertrauenspersonen dienen als Anlaufstelle für hilfesuchende
Personen, die Ausgrenzung, Diskriminierung oder Belästigung im Rahmen
der ZaPF erfahren haben.

Zu Beginn jeder ZaPF wählt das Plenum höchstens sechs
Vertrauenspersonen, zwei weitere Vertrauenspersonen werden von den
ausrichtenden Fachschaften benannt. Weiterhin sind die Personen des
Awarenessgremiums Vertrauenspersonen.

Unabhängig von der Zahl der zurückgetretenen Mitglieder gilt die
verbliebene Gruppe der Vertrauenspersonen weiterhin als gewählt.

Die Amtszeit der Vertrauenspersonen beginnt mit der Annahme der Wahl und
endet mit Beginn der nächsten ZaPF. Für vakante Plätze bei den
Vertrauenspersonen findet keine Nachwahl statt.

Die Wahl der Vertrauenspersonen ist in der Geschäftsordnung für Plenen
der ZaPF gesondert zu regeln.

\subsection{Das Awarenessgremium}\label{das-awarenessgremium}

Das Awarenessgremium koordiniert die Arbeit der Vertrauenspersonen über
mehrere ZaPFen hinweg, unterstützt die Vertrauenspersonen in ihrer
Arbeit und bereitet Veranstaltungen sowie Schulungen zum Thema Awareness
auf und zwischen den ZaPFen vor. Weiterhin unterstützt das
Awarenessgremium die ausrichtenden Fachschaften zum Thema Awareness.

Das Awarenessgremium besteht aus zwei Personen, die für ein Jahr gewählt
werden.

Die Amtszeit einer Person des Awarenessgremiums beginnt zu einer im
Sommersemester stattfindenden ZaPF, die der anderen zu einer im
Wintersemester stattfindenden ZaPF.

Die Personen des Awarenessgremiums gelten qua Amt als weitere
Vertrauenspersonen, zusätzlich zu den sechs vom Plenum gewählten
Vertrauenspersonen. Ihre Amtszeit als Vertrauenspersonen ist äquivalent
zu den Regelungen zur Amtszeit der Vertrauenspersonen.

\subsection{Das Kommunikationsgremium}\label{das-kommunikationsgremium}

Das Kommunikationsgremium ist ein gemeinsames Gremium von ZaPF und der
jDPG (junge Deutsche Physikalische Gesellschaft).

Die Aufgaben dieses Gremiums sind der Austausch zwischen ZaPF und jDPG
sowie die Unterstützung gemeinsamer Projekte.

Das Kommunikationsgremium besteht aus vier natürlichen Personen, wovon
von ZaPF und von jDPG jeweils zwei Mitglieder bestimmt werden.

Die ZaPF wählt ihre Mitglieder des Kommunikationsgremiums für eine
Amtszeit von einem Jahr. Die Amtszeit eines von der ZaPF gewählten
Mitglieds beginnt auf einer ZaPF im Sommersemester und die des anderen
Mitgliedes auf einer ZaPF im Wintersemester.

Die Details zur Zusammenarbeit werden durch ein Dokument zur
„Zusammenarbeit von ZaPF und jDPG`` festgehalten, welches nur durch
einen in gleicher Fassung verabschiedeten Beschluss von ZaPF und jDPG
geändert werden kann.

\subsection{Der Technische Organisationsausschuss aller
Physikfachschaften
(TOPF)}\label{der-technische-organisationsausschuss-aller-physikfachschaften-topf}

Der Technische Organisationsausschuss aller Physikfachschaften (TOPF)
ist für die Instandhaltung und Dokumentation der EDV-Projekte der ZaPF
verantwortlich.

Er besteht aus zwei vom Plenum zu bestimmenden Personen, die für die
Aufrechterhaltung des Betriebs und die Dokumentation der Basissysteme
hauptverantwortlich sind, und einer beliebigen Anzahl von freiwillig
Helfenden, die für die Dokumentation und den Betrieb von einzelnen
Projekten verantwortlich sind.

Die hauptverantwortlichen Personen sind dem Plenum und dem StAPF
rechenschaftspflichtig und an ihre Weisungen gebunden. Insbesondere hat
das Plenum die Möglichkeit, Datenschutzerklärungen und Nutzungsordnungen
sowohl für das Gesamtsystem als auch für einzelne Projekte zu bestimmen.

Die freiwillig Helfenden werden nicht gewählt, sondern durch die beiden
hauptverantwortlichen Personen gemeinsam bestimmt. Sie sind ihnen
rechenschaftspflichtig sowie an deren Weisungen und die erlassenen
Ordnungen gebunden.

Die Amtszeit einer hauptverantwortlichen Person beginnt zu einer im
Sommersemester stattfindenden ZaPF, die der anderen zu einer im
Wintersemester stattfindenden ZaPF.

Die Amtszeit der hauptverantwortlichen Personen beträgt ein Jahr.

\section{Satzungsänderungen}\label{satzungsuxe4nderungen}

Änderungen dieser Satzung benötigen eine Zweidrittelmehrheit, wobei
Beschlussfähigkeit des Plenums vor der Abstimmung zwingend festzustellen
ist. Satzungsänderungen sind nicht durch Initiativanträge möglich und
können nur auf dem Endplenum abgestimmt werden.

Wünsche nach einer Satzungsänderung sind bis spätestens sieben Tage vor
dem Anfangsplenum geeignet (z.B. über die ZaPF-Mailingliste) zusammen
mit einem Antragsentwurf oder mindestens einer schriftlichen Begründung
und einem konkreten Thema der Satzungsänderung anzukündigen.

Auf der ZaPF muss dann zwingend ein Arbeitskreis zum Thema der
vorgeschlagenen Satzungsänderungen durchgeführt werden, dessen
Satzungsänderungsantrag bzw. Satzungsänderungsanträge bis spätestens
15:00 Uhr am Vortag des Endplenums bei der die ZaPF ausrichtenden
Fachschaft eingereicht und ausgehängt werden müssen.

\section*{Schlussbestimmungen und
Änderungshistorie}\label{schlussbestimmungen-und-uxe4nderungshistorie}
\addcontentsline{toc}{section}{Schlussbestimmungen und
Änderungshistorie}

Die vorliegende Satzung wurde anlässlich der ZaPF '06 in Dresden
vorbereitet, mit einer Zweidrittelmehrheit der anwesenden Fachschaften
beschlossen und angenommen. Diese Satzung setzt alle bisherigen außer
Kraft. Sie trat zum 28.05.2006 in Kraft.

Inhaltliche Änderungen wurden vorgenommen auf der:

\begin{itemize}
\tightlist
\item
  Sommer-ZaPF 2007 in Berlin,
\item
  Sommer-ZaPF 2008 in Konstanz,
\item
  Sommer-ZaPF 2009 in Göttingen,
\item
  Sommer-ZaPF 2011 in Dresden,
\item
  Sommer-ZaPF 2013 in Jena,
\item
  Sommer-ZaPF 2014 in Düsseldorf,
\item
  Winter-ZaPF 2014 in Bremen,
\item
  Sommer-ZaPF 2015 in Aachen,
\item
  Winter-ZaPF 2015 in Frankfurt am Main,
\item
  Sommer-ZaPF 2016 in Konstanz,
\item
  Sommer-ZaPF 2017 in Berlin,
\item
  Winter-ZaPF 2018 in Würzburg,
\item
  Sommer-ZaPF 2019 in Bonn,
\item
  Winter-ZaPF 2019 in Freiburg,
\item
  Sommer-ZaPF 2023 in Berlin,
\item
  Winter-ZaPF 2023 in Düsseldorf,
\item
  und der Winter-ZaPF 2024 in Mainz.
\end{itemize}

\end{document}
